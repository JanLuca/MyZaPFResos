\documentclass[10pt,oneside]{scrartcl}

% Sprache und Encodings
\usepackage[english,ngerman]{babel}
\usepackage[T1]{fontenc}
\usepackage[utf8]{inputenc}
\usepackage{epigraph}

% Typographisch interessante Pakete
\usepackage{microtype} % Randkorrektur und andere Anpassungen

% References to Internet and within the document
\usepackage[pdftex,colorlinks=false,
pdftitle={Resolution zu Transparenz bei Drittmittelforschung},
pdfauthor={Martin Scheuch (FUB), Jan Luca Naumann (FUB)},
pdfcreator={pdflatex},
pdfdisplaydoctitle=true]{hyperref}

% Absaetze nicht Einruecken
\setlength{\parindent}{0pt}
\setlength{\parskip}{2pt}

% Formatierung auf A4 anpassen
\usepackage{geometry}
\geometry{paper=a4paper,left=20mm,right=20mm,top=10mm,bottom=10mm}

\renewcommand*\dictumwidth{0.95\linewidth}

% \hyphenation{anonymer ano-ny-mer}
\begin{document}

\section*{Resolution zu Transparenz bei Drittmittelforschung}

\textbf{Adressaten:} An alle deutschsprachigen Hochschulen und öffentlichen wissenschaftlichen Einrichtungen

\textbf{Antragssteller:} Martin Scheuch (FUB), Jan Luca Naumann (FUB)

\subsection*{Antrag:}
Die ZaPF möge beschließen:
\begin{quote}
Die ZaPF sieht die Bedeutung von Drittmitteln für die moderne Forschung an öffentlichen Einrichtungen, jedoch finden wir eine gewisse Transparenz bei der Durchführung von wissenschaftlichen Tätigkeiten für Dritte erstrebenswert. Deshalb fordert die ZaPF, dass bei Drittmittelprojekten folgende Angaben jährlich veröffentlicht werden:
\begin{enumerate}
\item Auftraggeber mit Sparte/Handlungsfeld der Abteilung (1)
\item Angaben der Geheimhaltungsvereinbarungen oder Publikationsbeschränkungen, u. a. Art, Dauer und Umfang
\item Titel (2)
\item Abstract bei Projektende (2)
\item Hochschule mit Organisationseinheit
\item Gesamtsumme
\item Projekt- und Vertragslaufzeit 
\end{enumerate}
~\\
Fußnote 1: Der Verwendungszweck der Forschungsergebnisse muss aus dem angegebenen Handlungsfeld hervorgehen.\\

Fußnote 2: In begründeten Fällen können zeitlich befristete Ausnahmen bis zu einer Höchstdauer von zwei Jahren zugelassen werden.\\

\underline{Begründung:}\\
Drittmittelforschung macht einen bedeutenen Teil der heutigen Arbeit an öffentlichen Forschunngseinrichtungen aus. Bei Projekten, die beispielsweise durch die DFG oder die EU gefördert werden, sind Transparenzrichtlinien zur Information der Öffentlichkeit bereits vorhanden. Es ist uns ein wichtiges Anliegen, dass diese Transparenz auf alle Bereiche der Drittmittelforschung erweitert wird. Dies wird als nötig angesehen, weil bei der Durchführung dieser wissenschaftlichen Forschung immer öffentliche Infrastruktur und Ressourcen mitgenutzt werden. In einigen Bundesländern Deutschlands gab und gibt es bereits Bestrebungen, eine Informationspflicht einzuführen. Insbesondere möchten wir bereits geschehene Umsetzungen wie im Landeshochschulgesetz Baden-Württembergs lobend erwähnen.\\
Uns ist klar, dass Unternehmen ein wirtschaftliches Interesse an den Ergebnissen der geförderten Forschung haben. Um den Unternehmen die nötige Zeit für die patentrechtliche Verwertung einzuräumen, haben wir eine Möglichkeit zur Schiebung der Veröffentlichung von Titel und Abstract bis zu maximal zwei Jahren in die Resolution aufgenommen.

\end{quote}

\end{document}
 

\documentclass[10pt,oneside]{scrartcl}

% Sprache und Encodings
\usepackage[ngerman]{babel}
\usepackage[T1]{fontenc}
\usepackage[utf8]{inputenc}
\usepackage{epigraph}

% Typographisch interessante Pakete
\usepackage{microtype} % Randkorrektur und andere Anpassungen

% References to Internet and within the document
%\usepackage[pdftex,colorlinks=false,
%pdftitle={Resolution für mehr Lebensqualität},
%pdfauthor={Björn Guth (Aachen), Jörg Behrmann (FUB), Wolfgang Bauer (Alter Sack) und der Rest des Git-Workshops},
%pdfcreator={pdflatex},
%pdfdisplaydoctitle=true]{hyperref}

% Absaetze nicht Einruecken
\setlength{\parindent}{0pt}
\setlength{\parskip}{2pt}

% Formatierung auf A4 anpassen
\usepackage{geometry}
\geometry{paper=a4paper,left=20mm,right=20mm,top=10mm,bottom=10mm}

\renewcommand*\dictumwidth{0.95\linewidth}

\hyphenation{anonymer ano-ny-mer}
\begin{document}

\section*{Resolution zur Umsetzung von elektronischen Studierendenausweisen}

\textbf{Adressaten:} An alle deutschsprachigen Hochschulen und Studentenwerke

\textbf{Antragssteller:} Markus Gleich (FUB), Jörg Germeroth (Uni Siegen), Hanna Kolkmann (Uni Rostock), \\ Leveke Holler (FUB), Jan Luca Naumann (FUB)

\subsection*{Antrag:}
Die ZaPF möge beschließen:
\begin{quote}
  In vielen Hochschulen wurden in den letzten Jahren elektronische Studierendenausweise eingeführt bzw. sind diese in Planung. Grundsätzlich begrüßen wir Vereinfachungen des Studierendenalltags, folgende Punkte  müssen jedoch beachtet werden:

  \begin{enumerate}
  \item{\underline{Chipkarte}: Es muss neben der Chipkarte auch eine Variante ohne jegliche elektronische Bauteile geben. 
  Für die elektronische Variante müssen folgende Standards gelten:}
    \begin{itemize}
      \item{Auf dem Chip dürfen nur die nötigsten Daten für die angeboten Funktionalitäten liegen. 
      Insbesondere dürfen keine persönlichen Daten wie Name, Adresse,  Geburtsdatum oder Matrikelnummer dort abgelegt werden.}
      \item{Alle Daten der einzelnen Funktionen müssen separat gespeichert, verschlüsselt und verarbeitet werden.}
      \item{Für die Chipkarte müssen aktuelle und sichere Verfahren verwendet werden. 
      Insbesondere eine nicht autorisierte Datenauslesung muss verhindert werden. Unserer Ansicht nach schließt dies kontaktlose Ausleseverfahren aus.} 
      \item{Bei Bekanntwerden von Sicherheitsmängeln müssen die Betroffenen umgehend informiert und das Problem behoben werden.}
  \end{itemize}
  \item{\underline{Matrikelnummer}: Wenn die Matrikelnummer nicht zwingend  auf dem Studierendenausweis für organisatorische Abläufe benötigt wird, bevorzugen wir, dass sie dort nicht aufgedruckt wird.}
  \item{\underline{Lichtbild}: Ein verpflichtendes Lichtbild für die Studierendenausweise wird abgelehnt. Ist ein Lichtbild vorgesehen, muss dieses optional sein.}
  \item{\underline{Bezahlfunktion/Mensa-Karte}:  Sollte eine Bezahlfunktion für Mensa, Kopierer, Drucker oder ähnliches verwendet werden, 
  muss diese anonym eingerichtet werden. Hierfür darf die ausgegebene bzw. verwendete ID an keiner Stelle mit den Personendaten verknüpft gespeichert werden, 
  so dass keine Rückführung auf den Besitzer der Karte möglich ist. 
  Insbesondere ist uns wichtig, dass keine Daten außer dem aktuellen Guthaben erfasst werden. Abbuchungen und Einkäufe dürfen somit nicht aufgezeichnet werden.}
  \item{\underline{Anwesenheitskontrolle}: Wir sind gegen jegliche Möglichkeit zur Kontrolle der Anwesenheit mit den Funktionen des Studierendenausweises. Dies gilt sowohl für Lehrveranstaltungen als auch für Tätigkeiten als studentische Hilfskraft. }
  \item{\underline{Prüfungsverwaltung}: Wir lehnen eine Verknüpfung des elektronischen Studierendenausweises und der Prüfungsverwaltung (An- und Abmeldung, Noteneinsicht) ab. Die Prüfungsverwaltung sollte unabhängig von den elektronischen Funktionen des Studierendenausweises durchführbar sein. Damit ist diese unabhängig von einem physischen Medium möglich, welches die Ausfallsicherheit erheblich erhöht. Dadurch ist die umständliche Implemetierung einer digitalen Signaturfunktion auf der Karte nicht notwendig.}
  \item{\underline{Semesterticket}: Der Nachweis der Beförderungsberechtigung sollte in den Studierendenausweis integriert werden. Dieser sollte ohne elektronische Komponente auskommen, um ein Tracking der Studierenden zu erschweren.}
  \item{\underline{Schlüsselkarte}: Wird der Ausweis als Schlüsselkarte für die Gebäude und Räume der Hochschule genutzt, muss die Protokollierung von Zutritten ausgeschlossen sein.}
  \item{\underline{Bibliothek}: Grundsätzlich befürworten wir eine Nutzung des Studierendenausweis auch als Bibliotheksausweis. Jedoch sollte für die Nutzung der Bibliothek (sowohl vor Ort als auch online) eine separate ID verwendet werden, die unabhängig von z.\,B. der Matrikelnummer und anderen Daten der Studierenden ist. Dabei sollte die ID nicht digital auf der Karte gespeichert sein.}
\end{enumerate}

\end{quote}

\end{document}
 

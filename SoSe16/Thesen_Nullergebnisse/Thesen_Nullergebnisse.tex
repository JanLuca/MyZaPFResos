\documentclass[DIV=9]{scrartcl}

% Sprache und Encodings
\usepackage[english,ngerman]{babel}
\usepackage[T1]{fontenc}
\usepackage[utf8]{inputenc}
\usepackage{epigraph}

% Typographisch interessante Pakete
\usepackage{microtype} % Randkorrektur und andere Anpassungen

% References to Internet and within the document
\usepackage[pdftex,colorlinks=false,
pdftitle={Resolution zu Veröffentlichungspflicht bei Drittmittelforschung},
pdfauthor={Martin Scheuch (FUB), Jan Luca Naumann (FUB)},
pdfcreator={pdflatex},
pdfdisplaydoctitle=true]{hyperref}

% Absaetze nicht Einruecken
\setlength{\parindent}{0pt}
\setlength{\parskip}{2pt}

% Formatierung auf A4 anpassen
\usepackage{geometry}
\geometry{paper=a4paper,left=20mm,right=20mm,top=10mm,bottom=10mm}

\renewcommand*\dictumwidth{0.95\linewidth}
\renewcommand{\thefootnote}{\arabic{footnote}}

% \hyphenation{anonymer ano-ny-mer}
\begin{document}

\section*{Thesenpapier zu Nullergebnissen}

%\textbf{Antragsteller:} Martin Scheuch (FUB), Jan Luca Naumann (FUB)

\subsection*{These 1: Definition Nullergebnisse}
Ein Nullergebnis erfüllt eins der folgenden Kriterien:
\begin{itemize}
\item Falsifizierung einer wissenschaftlichen Hypothese
\item Mehrdeutiges oder nicht beweiskräftiges Ergebnis
\item Nicht zielführendes Ergebnis auf dem Weg zur einer Veröffentlichung (``Trial \& Error'')
\end{itemize}
Dabei wurden ordentliche wissenschaftliche Standards bei Erlangung der Ergebnisse beachtet.

\subsection*{These 2: Wert von Nullergebnissen}
Nullergebnisse sind natürliche Begleiter ordentlicher Forschung. Allerdings sind sie nicht nur Abfallprodukte, sondern haben auch an sich einen wissenschaftlichen Wert, den es zu schätzen und zu wahren gilt. Es soll darauf hingewirkt werden, dass sie als Folge von gründlicher Arbeit gesehen werden.

\subsection*{These 3: Umgang mit Nullergebnissen}
Nullergebnisse sollen in ähnlicher Form zu normalen Resultaten wissenschaftlich aufbereitet und der Öffentlichkeit zur Verfügung gestellt werden.

\subsection*{These 4: Konzepte für Nullergebnisse}
Bei der Planung und Vorbereitung von wissenschaftlichen Projekten soll der Umgang mit Nullergebnissen thematisiert werden. Mögliche Konzepte sollen entsprechend der Art des Projekts dokumentiert werden.

\subsection*{These 5: Nullergebnisse als Projektbestandteil}
Nullergebnisse als normaler Bestandteil von wissenschaftlicher Arbeit sollen in angemessen Umfang bei Veröffentlichungen im Rahmen eines Projektes beachtet werden.

\subsection*{These 6: Zugänglichkeit von Nullergebnisse}
Es soll eine Plattform entstehen, um Nullergebnisse zu veröffentlichen, zu sammeln und in gebündelter Form zugänglich zu machen.

\subsection*{These 7: Veröffentlichungspflicht für Nullergebnisse}
Aus all diesen Gründen folgt, dass Nullergebnisse in geeigneter Form veröffentlicht werden müssen.

\end{document}
 

\documentclass[DIV=9]{scrartcl}

% Sprache und Encodings
\usepackage[english,ngerman]{babel}
\usepackage[T1]{fontenc}
\usepackage[utf8]{inputenc}
\usepackage{epigraph}

% Typographisch interessante Pakete
\usepackage{microtype} % Randkorrektur und andere Anpassungen

% References to Internet and within the document
\usepackage[pdftex,colorlinks=false,
pdftitle={Resolution zu Transparenz bei Drittmittelforschung},
pdfauthor={Martin Scheuch (FUB), Jan Luca Naumann (FUB)},
pdfcreator={pdflatex},
pdfdisplaydoctitle=true]{hyperref}

% Absaetze nicht Einruecken
\setlength{\parindent}{0pt}
\setlength{\parskip}{2pt}

% Formatierung auf A4 anpassen
\usepackage{geometry}
\geometry{paper=a4paper,left=20mm,right=20mm,top=10mm,bottom=10mm}

\renewcommand*\dictumwidth{0.95\linewidth}
\renewcommand{\thefootnote}{\arabic{footnote}}

% \hyphenation{anonymer ano-ny-mer}
\begin{document}

\section*{Resolution zu Transparenz bei Drittmittelforschung}

\textbf{Adressaten:} An alle deutschsprachigen Hochschulen, öffentlichen wissenschaftlichen Einrichtungen, die HRK, die KMK, die DPG, die KFP und das BMBF

\textbf{Antragsteller:} Martin Scheuch (FUB), Jan Luca Naumann (FUB)

\subsection*{Antrag:}
Die ZaPF möge beschließen:\\\\

Die ZaPF sieht die Bedeutung von Drittmitteln für die moderne Forschung an öffentlichen Einrichtungen. Auch wird der Gedanke, dass Forschung dem Allgemeinwohl dienen soll, als hoch erachtet. Deswegen fordert die ZaPF, dass die Ergebnisse von Drittmittelforschung nach einer Frist von zwei Jahren der Allgemeinheit in leichtzugänglicher Form zur Verfügung gestellt werden.\\

Als Ergebnisse, zu denen die Öffentlichkeit Zugang erhalten soll, sehen wir neben Abschlussarbeiten und Paper auch die Resultate von abgeschlossenen Forschungs(teil-)projekten wie z.B. einer abgeschlossenen Messreihe an.
~\\\\\\\\
\underline{Begründung:}\\\\
Drittmittelforschung macht einen bedeutenden Teil der heutigen Arbeit an öffentlichen Forschungseinrichtungen aus. Dadurch entsteht leider das Problem, dass bei industriegeförderter Forschung Ergebnisse, Daten und Abschlussarbeiten mit Sperrvermerken versehen werden, sodass die Öffentlichkeit keinen Zugriff darauf hat. Da jedoch bei der Durchführung von wissenschaftlichen Forschung an öffentlichen Einrichtungen immer öffentliche Infrastruktur und Ressourcen mitgenutzt werden, sehen wir es als notwendig an, dass die Allgemeinheit auch Zugriff auf Ergebnisse der durch sie unterstützten Forschung erhält.\\
Uns ist bewusst, dass Unternehmen ein wirtschaftliches Interesse an den Ergebnissen der geförderten Forschung haben. Um den Unternehmen die nötige Zeit für die patentrechtliche Verwertung zu geben sowie einen sinnvollen Zeitraum für die Vorbereitung einer Veröffentlichung zu ermöglichen, sehen wir die Notwendigkeit einer Frist von 2 Jahren ein.\\

\emph{Anm. zur Reso: Die Begründung soll mitveröffentlicht werden}


\end{document}
 

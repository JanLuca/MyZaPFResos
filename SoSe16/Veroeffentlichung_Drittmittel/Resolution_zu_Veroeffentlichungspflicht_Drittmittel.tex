\documentclass[DIV=9]{scrartcl}

% Sprache und Encodings
\usepackage[english,ngerman]{babel}
\usepackage[T1]{fontenc}
\usepackage[utf8]{inputenc}
\usepackage{epigraph}

% Typographisch interessante Pakete
\usepackage{microtype} % Randkorrektur und andere Anpassungen

% References to Internet and within the document
\usepackage[pdftex,colorlinks=false,
pdftitle={Resolution zu Veröffentlichungspflicht bei Drittmittelforschung},
pdfauthor={Martin Scheuch (FUB), Jan Luca Naumann (FUB)},
pdfcreator={pdflatex},
pdfdisplaydoctitle=true]{hyperref}

% Absaetze nicht Einruecken
\setlength{\parindent}{0pt}
\setlength{\parskip}{2pt}

% Formatierung auf A4 anpassen
\usepackage{geometry}
\geometry{paper=a4paper,left=20mm,right=20mm,top=10mm,bottom=10mm}

\renewcommand*\dictumwidth{0.95\linewidth}
\renewcommand{\thefootnote}{\arabic{footnote}}

% \hyphenation{anonymer ano-ny-mer}
\begin{document}

\section*{Resolution zu Veröffentlichungspflicht bei Drittmittelforschung}

\textbf{Adressaten:} An alle deutschsprachigen Hochschulen, öffentlichen wissenschaftlichen Einrichtungen, die HRK, die KMK, die DPG, die KFP, die DFG und das BMBF

\textbf{Antragsteller:} Martin Scheuch (FUB), Jan Luca Naumann (FUB)

\subsection*{Antrag:}
Die ZaPF möge beschließen:\\\\

Die ZaPF sieht die besondere Bedeutung von Drittmitteln für die Forschung an öffentlichen Einrichtungen. Auch wird der Gedanke, dass Forschung dem Allgemeinwohl dienen soll, als wichtig erachtet. Deswegen fordert die ZaPF, dass die Ergebnisse von Drittmittelforschung an öffentlich finanzierten Einrichtungen der Allgemeinheit in leichtzugänglicher Form zur Verfügung gestellt werden müssen. Als Ergebnisse, zu denen die Öffentlichkeit Zugang erhalten soll, sehen wir neben wissenschaftlichen Abschlussarbeiten (insbesondere Promotion und Habilitation) und Berichten auch die Resultate von abgeschlossenen Forschungsprojekten. Eine mögliche Sperrfrist muss zeitlich beschränkt sein. Wir empfehlen einen Zeitraum von zwei Jahren.

~\\\\\\\\
\underline{Begründung:}\\\\
Drittmittelforschung macht heute einen bedeutenden Teil der Arbeit an öffentlichen Forschungseinrichtungen aus. Es entsteht das Problem, dass Ergebnisse und Abschlussarbeiten bei industriegeförderter Forschung teils mit Sperrvermerken versehen werden. Dies hat zur Folge, dass die Öffentlichkeit keinen Zugriff darauf hat und Abschlussarbeiten als persönliche Leistung nicht verwendet werden können. Da bei der Durchführung von wissenschaftlicher Forschung an öffentlichen Einrichtungen immer staatlich finanzierte Infrastruktur und Ressourcen mitgenutzt werden, erachten wir es als notwendig, dass die Allgemeinheit auch Zugang zu den Ergebnissen der durch sie unterstützten Forschung erhält. Uns ist bewusst, dass Unternehmen ein wirtschaftliches Interesse an den Ergebnissen der geförderten Forschung haben. Um den Unternehmen die nötige Zeit für die wirtschaftliche Verwertung sowie für die Vorbereitung einer Veröffentlichung zu geben, erkennen wir die Notwendigkeit einer angemessenen Frist an.\\

\emph{Anm. zur Reso: Die Begründung soll mitveröffentlicht werden}


\end{document}
 

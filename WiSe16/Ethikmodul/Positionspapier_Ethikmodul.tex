\documentclass[DIV=9]{scrartcl}

% Sprache und Encodings
\usepackage[english,ngerman]{babel}
\usepackage[T1]{fontenc}
\usepackage[utf8]{inputenc}
\usepackage{epigraph}

% Typographisch interessante Pakete
\usepackage{microtype} % Randkorrektur und andere Anpassungen

% References to Internet and within the document
\usepackage[pdftex,colorlinks=false,
pdftitle={Resolution zu Veröffentlichungspflicht bei Drittmittelforschung},
pdfauthor={Martin Scheuch (FUB), Jan Luca Naumann (FUB)},
pdfcreator={pdflatex},
pdfdisplaydoctitle=true]{hyperref}

% Absaetze nicht Einruecken
\setlength{\parindent}{0pt}
\setlength{\parskip}{2pt}

\usepackage{draftwatermark}
\SetWatermarkText{Entwurf}
\SetWatermarkScale{2}

% Formatierung auf A4 anpassen
\usepackage{geometry}
\geometry{paper=a4paper,left=20mm,right=20mm,top=10mm,bottom=10mm}

\renewcommand*\dictumwidth{0.95\linewidth}
\renewcommand{\thefootnote}{\arabic{footnote}}

% \hyphenation{anonymer ano-ny-mer}
\begin{document}

\section*{Positionspapier zu Ethikinhalten im Physikstudium}

\textbf{Antragsteller:} Leonhard Günther (TU München), Jennifer Hartfiel (FUB), Jan Luca Naumann (FUB)

\subsection*{Antrag:}
Die ZaPF möge beschließen:\\\\

\underline{Vorschlag 1:} Die ZaPF spricht sich dafür aus, Konzepte zu entwickeln, welche Ethik im Physikstudium präsenter machen.\\\\

\underline{Vorschlag 2:} Die ZaPF spricht sich dafür aus, Ethikinhalte in einem angemessen Umfang im Physikstudium anzubieten, sodass die Möglichkeit geboten wird, sich auch im Rahmen des Studiums mit ethischen Fragenstellungen auseinanderzusetzen.

~\\\\\\\\
\underline{Begründung:}\\\\
Im AK Ethikmodul wurde nach langer Diskussion festgestellt, dass alle Beteiligten sich darin einig sind, dass Ethikinhalte in einem Physikstudium einen angemessen Rahmen erhalten sollen. Konkrete Ausgestaltungsvorschläge sind schwierig allgemein festzuhalten, da viel von der lokalen Situation der Universitäten abhängt wie z.B. Möglichkeiten zur Kooperation mit Philosophiefachbereichen oder die Umsetzung in den einzelnen Studiengängen. 


\end{document}
 
